\documentclass{article}
\usepackage[utf8]{inputenc}
\usepackage{booktabs,siunitx}
\usepackage{xcolor}
\usepackage[english]{babel}
\usepackage[utf8]{inputenc}

\title{Dictionary TLA++ to Cubicle language}


\begin{document}
\large

\sisetup{output-decimal-marker={,}}

 \subsection*{Observations} 
The following are some thoughts about the strategy of translating a TLA+ Spec into a Cubicle one. 
 Some of the following ideas are inspired from \cite{10.1007/978-3-642-30729-4_3} and \cite{merz:hal-00760570})

\paragraph{Data Types.} For the basic data types (integer, Boolean and string), the translation is straightforward. 

\underline{Tuples:} For one-dimensional case, tuples  can be translated into arrays. I am not sure how to translate a condition such as "Len(Tuple)=5". A  naive way can be that we associate every tuple defined in the TLA+ spec to an integer variable (in Cubicle) that declares its length. I still do not see a better way.

For arrays with higher dimensions, I still do not know. 

%BigTuple== << Tuple1, Tuple2, Tuple3 >>


%type  abstract_tuple = ab_Tuple1 | ab_Tuple2 |ab_Tuple3
%var BigTuple  :   abstract_tuple
%array Tuple1 : int
%array Tuple2 : int 
%array Tuple3 : int  




\paragraph{Variable declaration.}    TLA+ is an untyped, while  the language  of Cubicle is. \color{red}My initial thoughts were that the information in the "Init"  state, the constants and the "ASSUME" statement should be enough to get the type of a variable (with the help of Ocaml). \color{black}However, it is not clear (to me) how feasible this way is. 

So, probably an approach  similar to \cite[Section 3]{merz:hal-00760570} is more promising, where an inference algorithm is presented.   



\vspace{2 cm}
 \subsection*{Straightforward (or almost) translation} 

\begin{tabular}{ |p{3cm}|p{3cm}|  }
 
 \hline
 \multicolumn{2}{|c|}{ \textbf{Logic}}  \\
 \hline

 \textbf{TLA++} & \textbf{ Cubicle}  \\
 \hline
  \hline
  TRUE &  True\\ 
 FALSE &  False\\ 
 BOOLEAN &  bool\\ 

 \textbackslash/ & $ || $ \\ 
  /\textbackslash & $\&\&$ \\ 
  \hline

\end{tabular}

\vspace{2 cm}

\begin{tabular}{ |p{3cm}|p{3cm}|  }
 
 \hline
 \multicolumn{2}{|c|}{ \textbf{Data type}}  \\
 \hline

 \textbf{TLA++} & \textbf{ Cubicle}  \\
 \hline
  \hline
  CONSTANTS &  const\\ 
  \hline

\end{tabular}


\bibliographystyle{unsrt}
\bibliography{bibl}

\end{document}

The following table is dedicated to make a comparison between terms in TLA++ and Cubicle's  input language. 

\vspace{1 cm	}


\begin{tabular}{ |p{3cm}||p{3cm}|p{7cm}|  }
 \hline

 \textbf{TLA++} & \textbf{ Cubicle}  & \textbf{ Comments }\\
 \hline
  \hline
 \multicolumn{3}{|c|}{ \textbf{Variable declaration}}	 \\
 \hline
   Five == 5  &  var Five : 5   &\\

   constants &     &\\
 \hline
 \multicolumn{3}{|c|}{ \textbf{Operators} } \\
 \hline

   &     &\\

    &     &\\
  \hline

   \multicolumn{3}{|c|}{ \textbf{Expressions} } \\
 \hline

   &     &\\

    &     &\\
  \hline
\end{tabular}


